%\documentclass[a4paper,10pt]{article}
\documentclass[letterpaper, 11pt, conference]{ieeeconf}
\usepackage{amsmath} % For 'cases' curly brace
\usepackage{amsfonts} % For bold number set font
\usepackage{color}
\usepackage{algorithmic}
\usepackage{algorithm}
\usepackage{amssymb}
\usepackage{url}

\newcommand{\xxx}[1]{\textcolor{red}{#1}}

\begin{document}
\title{Validating Cell Tower Clustering for Mobility Prediction}
\author{Mehrdad Moradi\\moradi@umich.edu \and Pedro d'Aquino\\pdaquino@umich.edu}
\date{December 11, 2012}

\maketitle

\xxx{Abstract}

\section{Introduction}
\label{sec:intro}
\xxx{Talk about mobility prediction in mobile networks. Discuss some of the potential benefits -- some of them were discussed in the
ATT leap graph paper. Maybe do a little research on this.}

In this project, we address one of the challenges faced in mobility prediction using GSM cell tower connection logs. In our context, mobility prediction in cellular
towers can be defined as the ability to predict a user's physical movement by examning the history of cell tower connections. This is very important
for mobile network operators, as several network optimizations are possible if they can forecast with reasonable accuracy
where a user will be in the future \cite{LeapGraph} \cite{Liu97anoptimal}. By being able to pre-allocate resources to towers that are most likely to
be connected to by the user, network operators can improve latency and uptime. This is increasingly important as the amount of data in cellular networks
increases: indeed, in 2010 the amount of data traffic surpassed voice traffic for the first time, and it is expected to double annually in the near future
\cite{EricssonData}.

Another use of mobility prediction is to discover mobility profiles, i.e. patterns in a user's movement \cite{mobilityprofiler}. As illustrated in \cite{mobilityprofiler}, mobility profiles have a wide range of possible applications, from traffic planning to advertisement. Recently, Google started integrating
a contextual search and recommendation service in its phones that is able to, for instance, detect a user's home and work addresses and suggest routes based on
traffic \cite{googleNow}.


The heart of every mobility prediction system based on cell tower connections is to infer movement based on which cell the user's phone is connected to. The goal
is to predict which tower the user is more likely to connect to next. A major challenge in this is dealing with the \textit{oscillation effect}, which happens when a \textit{stationary} user's phone switches between two or more towers. This is a natural consequence of the radio medium and the fact that towers very often have overlapping ranges. In order to accurately predict a user's mobility, these outliers have to be removed so that fluctuations can be separated from actual user movement.

In this project, we analyze one published algorithm for oscillation detection \cite{mobilityprofiler} and evaluate its performance using GPS data as ground-truth. We find that this algorithm performs surprisingly well as measured by two metrics we define.

This report is structured as follows. In the next section, we motivate our work by explaining why clustering validation is important. In section~\ref{sec:methodology} we describe our methodology  and explain the algorithm we will evaluate. We present and discuss results in section~\ref{sec:results}, and conclude in section~\ref{sec:conclusion}.
\xxx{More about the results.}

\section{Motivation}

\xxx{This is where we explain why we think that our work is important: because current tower clustering methods do not have ground truth data to check against and we do.}

\xxx{How do we cite the Nokia MDC?}

\xxx{Reproduce the stationary phone example in the leap graph paper.}

The accuracy of current mobility prediction systems based on cell tower connections is usually measured through machine learning techniques like cross-validation that try to validate the ultimate goal of forecasting the user's future connections \cite{LeapGraph}. As far as we know there are no studies that evaluate methods of dealing with the oscillation effect, even though all prediction systems must deal with it.

In this paper we report the results of such an analysis. We have implemented and evaluated the ``oscillation graph'' technique described in \cite{mobilityprofiler}. The basic idea of this algorithm is to detect cell towers that often present oscillations and consider them as a single tower. This is effectively a clustering algorithm. Our goal is to evaluate whether towers that are clustered together are actually close in real life. To evaluate this, we use the Nokia Mobility Challenge dataset, which has data on both cell tower connection (``GSM logs'') and the GPS position of the phones.

\section{Methodology}
\label{sec:methodology}
\xxx{What were our methods?\\- Implemented clustering algorithm. Explain mobility paths, oscillations, oscillation graph etc. \\- RM. We sampled the RM dataset and compared it to
the Nokia MDC \\- MDC. We apply the oscillation clustering to the MDC and look at the position
characteristics of the clusters.}

The Mobility Profiler paper \cite{mobilityprofiler} describes an algorithm to detect the oscillation effect. Our approach to evaluating this algorithm's efficiency was the following. We implement the algorithm as described in \cite{mobilityprofiler} and test it in the same dataset used in the original paper. However, that dataset does not have any data on the location of the users when they were connected to the towers. We therefore apply the algorithm to another dataset, from the Nokia Mobility Challenge, in which the GPS position of the users is listed. In order to evaluate how successful the clustering algorithm is, we devise two metrics: cluster distance and cluster incompleteness.

\subsection{Clustering algorithm}

There are two main papers that tackle the problem of oscillation effects: the Mobility Profiler paper, by Bayir et al. \cite{mobilityprofiler}, and the Leap Graph paper by Duffield et al \cite{LeapGraph}. Ideally we would want to evaluate both of them, but the leap graph algorithm, which was published by resarchers at AT\&T, makes use of the active set of connections. The active set contains all towers that are within range of the phone's antenna at any given time. Unfortunately, active set logs are not public which makes it impossible for us to evaluate the leap graph algorithm.

We therefore focus our efforts on the algorithm described in the Mobility Profiler paper, which we call the oscillation graph algorithm. In the interest of completeness, we briefly describe how it works.

\xxx{Cellspan log example. Mobility path example.}

\subsubsection{Mobility paths}
The input of the algorithm is a sequence of \textit{handovers}, which are transition events when a phone switches the tower it is connected to. These events are timestamped. The first step of the algorithm is to analyze the transitions and extract \textit{mobility paths}.  A mobility path corresponds to a sequence of handovers that are associated with actual user movement. For example, imagine a user that drives every morning to work, stays there for eight hours and goes back home. During his drive to work, the user's phone will switch cell towers several times. These handovers will be included in a mobility path. While the user is at work, ignoring the oscillation effect temporarily, the phone will be connected to a single tower. Finally, when the user the goes home the handovers that occur during the drive will belong to a different mobility path.

More formally, let $\Delta_i$ be the time difference between two consecutive handovers for a particular user's connection log. In other words, $\Delta_i$ is the time the user spent connected to cell tower $C_i$. Let $\delta_{duration}$ be the \textit{end-location threshold}. Then we define a mobility path as a sequence of cell connections $\left\{C_i, C_{i+1}, \ldots, C_j\right\}$ such that $\Delta_i < \delta_{duration}$ for all handovers.\footnote{We also have to consider \textit{hidden end-locations}, which happen when a user's phone is disconnected from any towers, or if the user turns his phone off. We omit its formal definition for briefness.}


\subsubsection{Oscillating pair detection}

The goal of the oscillation detection algorithm is to retrieve sets of towers that overlap each other -- that is, towers that could be in a stationary user's handover logs. The main idea behind the algorithm is that a user's phone is more likely to switch between oscillating towers than between regular towers that are not overlapping. In this context, \textit{switching} means a handover from one tower to another inside of a mobility path. We can define $k$ to be the minimum switching count and consider all pairs of towers that have more than $k$ switches in one mobility path to be an \textit{oscillating pair}. The example in \cite{mobilityprofiler} is illustrative: consider a sequence of handovers $\left[x, y, x, w, v, w, y\right]$ where every element represents one cell tower. We say that the pair $(x, y)$ has 3 switches: the first twofrom $x$ to $y$ and back to $x$ in the first positions of the sequence, and then back to $y$ at the end. Hence, if we adopt $k=3$ the only oscillating pair would be $(x, y)$. Note that switches do not need to be consecutive. The authors argue that this allows for detection of overlapping towers in dense region where multiple towers may be within range at the same time.

\subsubsection{Oscillation graph}

After detecting all the oscillating pairs in all mobility paths, the next step is to build the oscillation graph. The set $V$ of vertices in the graph are all cell towers in the dataset, and the weighted edges in $E$ connect oscillating pairs. Let $P_i$ be a mobility path.The weight of every edge, which we also call its \textit{support}, is defined as:

\begin{equation*}
s(x,y) = \frac{|\left\{P_i| (x, y) \in P_i \wedge (x, y) \text{oscillated}\right\}|}{|\left\{P_i| (x, y) \in P_i\right\}|}
\end{equation*}

In words, $s(x,y)$ is the ratio of how many times the cell towers $x$ and $y$ oscillated over the number of mobility paths in which $x$ and $y$ appear. For example, a pair of towers such that in \textit{every} mobility path that both of them appear the switch count is greater than $k$ would have a support equal to 1. If a pair only switches more or equal to $k$ times in half the mobility paths both of the towers appear in, that pair's support would be 0.5.

\subsubsection{Oscillation graph clustering}

\xxx{Explain the clustering algorithm.}

The final step of the algorithm is to find clusters in the oscillation graph. Intuitively, a cluster represents a group of towers that are not indicative of actual user movement. The Mobility Profiler paper uses an greedy divise algorithm for clustering. At every iteration, the algorithm removes the edge with the lowest weight from the graph. It then evaluates the ``quality'' of all connected components, which is the weighted ratio of edges inside of the cluster over the number of edges that leave the cluster in the original graph. Formally, let $C$ be a cluster with the set of edges $E$, and $E_{out}$ be the set of edges that have exactly one vertex in $C$, and define $w(e)$ to be an edge's weight:

\begin{equation*}
Q(C) = \frac{\displaystyle\sum_{\forall e_{in} \in E}{w(e_{in})}}{\displaystyle\sum_{\forall e_{out} \in E_{out}}{w(e_{out})}}
\end{equation*}

Every cluster with quality metric greater than a threshold is removed from the graph. The algorithm continues until there are no more vertices in the graph. The algorithm outputs all the clusters it found. Our main goal is to evaluate these clusters according to the GPS position of the users when they were within range of that tower.

\subsection{GSM datasets}

\xxx{Reality Mining dataset reference.}

We use two datasets with GSM cell tower connection data, from the Reality Mining project and from the Nokia Mobility Data Challenge.

\subsubsection{Reality Mining}

\xxx{Describe the reality mining dataset}

\subsubsection{Nokia Mobility Data Challenge -- GSM}

\xxx{Describe the Nokia Mobility Challenge}

\subsection{Cell location datasets}

\xxx{GPS logs from MDC and OpenCell}

Our main contribution is an analysis of the oscillation clustering algorithm using GPS data. We use two sources of GPS data, the Nokia MDC and the OpenCell database.

\subsubsection{Nokia Mobility Data Challenge -- GPS}

Phones in the MDC also had their GPS units sampled every minute. The format of the GPS logs is shown in Figure \xxx{Show the MDC GPS logs}. We process the GPS and try to establish an approximate position for each cell tower. We approximate the cell tower position as the average of the positions of users when they were connected to that tower. To do this, we have to match each entry in the GSM log with an entry in the GPS log.

Because GSM and GPS entries do not necessarily have the same timestamp (in fact, they almost never do), we use GPS entries that have a timestamp within $\delta_{GPS}$ seconds of the GSM entry we are trying to match. The algorithm goes through every record in the GSM log and tries to find a matching GPS log. After going through all users' connection logs, we estimate each tower's position as the average of all GPS positions that were associated with that tower. It is important to have a measure of how accurate the GPS estimate of a particular tower is, so we record the number of GPS samples for that tower and the standard deviation in the samples. The standard deviation is computed based on the distance from each sample to the average point, which is calculated using the Haversine formula \cite{haversine}.

The algorithm runs in $O(k(n\log n + m\log n))$, where $k$ is the number of users in the dataset and $m$ and $n$ are upper bounds on the number of GPS and GSM records per user, respectively. The logarithmic factor comes from sorting the GPS log by timestamp and performing a binary search.

\subsubsection{OpenCell}

\xxx{Talk about OpenCell; what it is and how we use it.}

\section{Results}
\label{sec:results}
\xxx{Results introduction fluff.}

\xxx{Talk about challenges in handling the data -- wrong clock, entries out of order in the RM dataset;}

We implemented the clustering algorithm in \cite{mobilityprofiler} for the Reality Mining and the Nokia Mobility Data Challenge datasets and compared the results of the clustering in both datasets (adjusting for the fact that the MDC dataset's data is sampled every minute). In this section, we report the results of running the algorithm in those datasets and analyze the results. \xxx{One sentence description of the non-GPS related results.} We also used GPS data to evaluate the accuracy and quality of the resulting clusters. \xxx{One sentence description of the GPS results}.

Running the clustering algorithm on the RM dataset was more challenging than originally expected; there were inconsistencies with the dates (apparently the phones' clocks reverted to 01/01/2004 when the battery died) and time (with entries appearing in an inconsistent order; see \xxx{RM GSM time weirdness example}). We dealt with these problems by ignoring records with a timestamp equal to 01/01/2004 and by assuming that the ordering of the lines in the dataset logs represented the actual ordering that the logs were collected -- this seemed consistent with manual analysis of the logs and generated valid mobility paths, which is not the case if the timestamps are taken at face value. \xxx{Maybe expand on what a valid mobility path is.}

In all experiments we used the same parameters value as the original Mobility Profiler paper, namely \xxx{list the constants we used, like the minimum oscillation count etc.}.

We are only able to assess the cluster distance and incompleteness in the MDC dataset, since it is the only one for which we have GPS information\footnote{In principle we could have used OpenCell data to estimate the position of the towers in the RM dataset. However, the data does not provide the necessary information about the cell towers, such as network ID.}. We choose the MDC GPS logs as our source of location data for the cell towers for two reasons. First, as discussed above, there are more cells in the MDC dataset covered by the GPS logs than by the OpenCell database. Second, because the GPS logs measure the \textit{user's} location when he or she was connected to that tower, it provides a semantically more accurate value -- after all, as discussed in Section \ref{sec:intro}, the ultimate goal of the clustering algorithm is to provide data that will be fed to an algorithm that predicts \textit{user's} movements.

\subsection{Dataset comparison}


\xxx{Compare MDC and Reality Mining, i.e. show that they are similiar using the sampled RM?}

We compare the Reality Mining dataset, which was originally used in the Mobility Profiler paper, with the Mobility Data Challenge. Table \xxx{Table with number of cells etc.} shows the number of cells, area codes and countries that are represented in each dataset. We can see that the MDC data has a much larger breadth in terms of the countries that are covered, but contains less cells and cell transitions. \xxx{Try to explain this difference. Talk about number of clusters.}

However, both datasets show similar patterns in cluster size distribution, as evidenced in Figure \xxx{Figure with the oscillation histogram}, which also presents the cluster size distribution for the Reality Mining dataset sampled every minute. This is necessary for a fair comparison since the RM contains information on every handover, while the MDC samples the tower the phone is currently connected to every minute (so there is loss of information).

The fact that both datasets show a similar distribution is important for our results, since it allows us to extrapolate the results we find with the MDC dataset to the RM dataset with reasonable confidence.

\subsection{Cluster statistics}

\xxx{Talk about cluster size, quality metric for both datasets}

\subsection{Location statistics}

\xxx{Talk about how many cells we have location data for, the difference between GPS logs and OpenCell, how many cells with $gps_{sightings} > 1$ and $stdddev < x$ etc.}

\subsection{Effectiveness of clustering algorithm}

In this section we present the most important results of our project: an empirical validation of the clustering algorithm from the point of view of the efficiency of the clusters that it outputs.

\subsubsection{Metrics}

We define two metrics, cluster distance and cluster incompleteness.

\paragraph{Cluster Distance} the average distance between cells in a cluster. A good algorithm will group together towers that are geographically close and therefore clusters will have a low distance. The cluster distance is measured in meters.

\paragraph{Cluster incompleteness} For a cell $k$, this is the number of cells within $\delta_{incomp}$ meters of it that are not in the same cluster as $k$. The algorithm is shown in \xxx{Show the cluster incompleteness algorithm.} Intuitively this measures if there are potentially oscillating towers that are not being clustered together. We can think of the cluster incompleteness as the dual of the cluster distance: the former favors agressive algorithms while the latter encourages conservative approaches. Another way to see the benefit of using two metrics is noting that there are trivial solutions that minimize distance or incompleteness (each cell in its own cluster and one huge cluster with all cels, respectively), but there are no trivial solutions that minimize both metrics at the same time.

\xxx{Talk about\begin{itemize}
	\item Cluster distance histogram/CDF, average.
	\item Mean cluster distance/cluster size (no correlation).
	\item Cluster incompleteness CDF, average.
	\item What can we conclude from this? That the algorithm works, pretty much.
\end{itemize}}

\subsubsection{Cluster distance}

In order to provide some confidence in the precision of the results, we only consider cells that were seen more than once in the GPS logs, with a standard deviation less than 500 meters. Only clusters in which more tha 70\% of cells meet this criteria are analyzed. This meant that only 82 clusters, or 9.44\% of the total, could be analyzed. Although this is a small fraction, there is no reason to think it is a biased sample if we assume there is no correlation between GPS precision and clustering ineffiency. Therefore, the results should hold for the entirety of the clusters, and, we conjecture, for other datasets too.

Figures \xxx{cluster distance histogram} and \xxx{cluster distance CDF} show the histogram and CDF of the cluster distances. \xxx{Talk about that.}

We also investigated whether there is a correlation between the size of a cluster and the its distance. A graph of both values is presented in Figure \xxx{size x distance}. We find that there is no significant correlation.

\subsubsection{Cluster incompleteness}

The measurements for cluster incompleteness are presented in Figures \xxx{cluster incompleteness histogram} and \xxx{cluster incompleteness CDF}. The average incompleteness for all cells with GPS data is 0.52 (standard deviation 1.37). This is a surprisingly low value and indicates that the algorithm works well.

\section{Conclusion}
\label{sec:conclusion}
We have evaluated a clustering algorithm used to eliminate the oscillation effect in cellular mobility prediction. Our main contribution is the use of GPS data to evaluate if the clustered towers are geographically close to each other, which would be circumstantial evidence in favor of the clustering algorithm. We find that the clustering algorithm proposed in \cite{mobilityprofiler} performs well on the two metrics we define, cluster distance and cluster incompleteness.

As part of future work, we would like to repeat the experiments here with more data. As discussed before, only 9.44\% of the clusters in the MDC dataset had enough information to be reliably processed by the algorithm. Another avenue of research is to investigate clustering algorithms that use a different clustering quality metric. One possible approach would be the modularity metric defined by Newman and Girvan~\cite{newman}. There is also a large body of literature on clustering in theoretical computer science, especially in social networks: Andersen and Peres find sparse cuts efficently in~\cite{sparsecuts}, while Aroroa et. al try to find overlapping communities \cite{overlapping}. Finding overlapping communities has interesting applications for dealing with the oscillation effect, since towers can conceivably in more than one cluster.

\bibliographystyle{IEEEtranS}
\bibliography{eecs589Report}

\end{document}
