\documentclass[a4paper,12pt]{article}
\usepackage{amsmath} % For 'cases' curly brace
\usepackage{amsfonts} % For bold number set font
\usepackage{color}
\usepackage{algorithmic}
\usepackage{algorithm}
\usepackage{amssymb}

\newcommand{\xxx}[1]{\textcolor{red}{#1}}

\begin{document}
\title{Validating Cell Tower Clustering for Mobility Prediction}
\author{Mehrdad Moradi\\moradi@umich.edu \and Pedro d'Aquino\\pdaquino@umich.edu}
\date{December 11, 2012}

\maketitle

\section{Introduction}

\xxx{Talk about mobility prediction in mobile networks. Discuss some of the potential benefits -- some of them were discussed in the
ATT leap graph paper. Maybe do a little research on this.}

\cite{Jadbabaie2012}
\cite{DeGroot1974}

\section{Motivation}

\xxx{This is where we explain why we think that our work is important: because current tower clustering methods do not have ground truth data to check against and we do.}

\section{Methodology}

\xxx{What were out methods?}

\section{Clustering algorithm}

\xxx{Explain why we used the Mobility Profiler algorithm; mention that we couldn't use a leap graph because we don't have active set information.
\\Describe the clustering algorithm.}

\subsection{GSM datasets}

\xxx{MDC and Reality Mining.}

\subsection{Cell location datasets}

\xxx{GPS logs from MDC and OpenCell}

\section{Results}

\subsection{Dataset comparison}

\xxx{Compare MDC and Reality Mining, i.e. show that they are similiar using the sampled RM?}

\subsection{Location comparison}

\bibliographystyle{plain}
\bibliography{eecs589Report}

\end{document}
